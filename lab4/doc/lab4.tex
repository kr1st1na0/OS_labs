\documentclass[a4paper, 12pt]{article}
\usepackage{cmap}
\usepackage[12pt]{extsizes}			
\usepackage{mathtext} 				
\usepackage[T2A]{fontenc}			
\usepackage[utf8]{inputenc}			
\usepackage[english,russian]{babel}
\usepackage{setspace}
\singlespacing
\usepackage{amsmath,amsfonts,amssymb,amsthm,mathtools}
\usepackage{fancyhdr}
\usepackage{soulutf8}
\usepackage{euscript}
\usepackage{mathrsfs}
\usepackage{listings}

\usepackage[colorlinks=true, urlcolor=blue, linkcolor=red]{hyperref}

\pagestyle{fancy}
\usepackage{indentfirst}
\usepackage[top=10mm]{geometry}
\rhead{}
\lhead{}
\renewcommand{\headrulewidth}{0mm}
\usepackage{tocloft}
\renewcommand{\cftsecleader}{\cftdotfill{\cftdotsep}}
\usepackage[dvipsnames]{xcolor}

\lstdefinestyle{mystyle}{ 
	keywordstyle=\color{OliveGreen},
	numberstyle=\tiny\color{Gray},
	stringstyle=\color{BurntOrange},
	basicstyle=\ttfamily\footnotesize,
	breakatwhitespace=false,         
	breaklines=true,                 
	captionpos=b,                    
	keepspaces=true,                 
	numbers=left,                    
	numbersep=5pt,                  
	showspaces=false,                
	showstringspaces=false,
	showtabs=false,                  
	tabsize=2
}

\lstset{style=mystyle}

\begin{document}
\thispagestyle{empty}	
\begin{center}
	Московский авиационный институт
	
	(Национальный исследовательский университет)
	
	Факультет "Информационные технологии и прикладная математика"
	
	Кафедра "Вычислительная математика и программирование"
	
\end{center}
\vspace{40ex}
\begin{center}
	\textbf{\large{Лабораторная работа №4 по курсу\linebreak \textquotedblleft Операционные системы\textquotedblright}}
\end{center}
\vspace{35ex}
\begin{flushright}
	\textit{Студент: } Былькова Кристина Алексеевна
	
	\vspace{2ex}
	\textit{Группа: } М8О-208Б-22
	
	\vspace{2ex}
	\textit{Преподаватель: } Миронов Евгений Сергеевич
	
	\vspace{2ex}
	\textit{Вариант: } 2 
	
	\vspace{2ex}
	\textit{Оценка: } \underline{\quad\quad\quad\quad\quad\quad}
	
	 \vspace{2ex}
	\textit{Дата: } \underline{\quad\quad\quad\quad\quad\quad}
	
	\vspace{2ex}
	\textit{Подпись: } \underline{\quad\quad\quad\quad\quad\quad}
	
\end{flushright}

\vspace{5ex}

\begin{vfill}
	\begin{center}
		Москва, 2023
	\end{center}	
\end{vfill}
\newpage

\begingroup
\color{black}
\tableofcontents\newpage
\endgroup

\section{Репозиторий}
\href{https://github.com/kr1st1na0/OS\_labs}{https://github.com/kr1st1na0/OS\_labs}

\section{Цель работы}
Приобретение практических навыков в:
\begin{itemize}
  \item Создании динамических библиотек
  \item Создании программ, которые используют функции динамических библиотек
\end{itemize}

\section{Задание}
Требуется создать динамические библиотеки, которые реализуют определенный функционал. Далее использовать данные библиотеки 2-мя способами:
\begin{enumerate}
  \item Во время компиляции (на этапе «линковки»/linking)
  \item Во время исполнения программы. Библиотеки загружаются в память с помощью интерфейса ОС для работы с динамическими библиотеками
\end{enumerate}
В конечном итоге, в лабораторной работе необходимо получить следующие части:
\begin{itemize}
  \item Динамические библиотеки, реализующие контракты, которые заданы вариантом;
  \item Тестовая программа (программа №1), которая используют одну из библиотек, используя знания полученные на этапе компиляции;
  \item Тестовая программа (программа №2), которая загружает библиотеки, используя только их местоположение и контракты.
\end{itemize}

\section{Описание работы программы}
Функции, написанные в результате выполнения лабораторной работы:
\begin{itemize}
  \item Рассчет производной функции cos(x) в точке A с приращением deltaX двумя реализациями
  \item Подсчёт количества простых чисел на отрезке [A, B] (A, B - натуральные) наивным алгоритмом и решетом Эратосфена
\end{itemize}

В ходе выполнения лабораторной работы я использовала следующие системные вызовы:
\begin{itemize}
  \item dlopen - открытие динамического объекта
  \item dlclose - закрытие динамического объекта
\end{itemize}


\newpage

\section{Исходный код}
utils.hpp
\begin{lstlisting}[language=C++]
#include <iostream>
#include <cstdlib>
#include <dlfcn.h>

using SinIntegralFunc = float (*)(float, float, float);
using PrimeCountFunc = int (*)(int, int);

void* LoadLibrary(const char *libraryName);
void UnloadLibrary(void* handle);
\end{lstlisting}

realizations.hpp
\begin{lstlisting}[language=C++]
#pragma once

#include <iostream>
#include <cmath>

#ifdef __cplusplus
extern "C" {
#endif

constexpr int NUM_POINTS = 3000; // число разбиений
const float PI = 3.1415926535;

float SinIntegral(float a, float b, float e);
int PrimeCount(int a, int b);

#ifdef __cplusplus
}
#endif
\end{lstlisting}

utils.cpp
\begin{lstlisting}[language=C++]
#include "utils.hpp"

void* LoadLibrary(const char *libraryName) {
    void* handle = dlopen(libraryName, RTLD_LAZY);
    if (!handle) {
        std::cerr << "Couldn't load the library: " << dlerror() << std::endl;
        exit(EXIT_FAILURE);
    }
    return handle;
}

void UnloadLibrary(void* handle) {
    if (dlclose(handle) != 0) {
        std::cerr << "Couldn't unload the library: " << dlerror() << std::endl;
        exit(EXIT_FAILURE);
    }
}
\end{lstlisting}

realization1.cpp
\begin{lstlisting}[language=C++]
#include "realizations.hpp"

extern "C" float SinIntegral(float a, float b, float e) {
    float integral = 0.0;
    e = (b - a) / NUM_POINTS;
    for (int i = 0; i <= NUM_POINTS; ++i) {
        integral = integral + e * sin(a + e * (i - 0.5));
    }
    return integral;
}

extern "C" int PrimeCount(int a, int b) {
    int count = 0;
    bool flag = true;
    for (int i = a; i <= b; ++i) {
        if (i <= 1) {
            continue;
        }
        for (int j = 2; j < i; ++j) {
            if (i % j == 0) {
                flag = false;
                break;
            }
        }
        if (flag) {
            ++count;
        }
        flag = true;
    }
    return count;
}
\end{lstlisting}

realization2.cpp
\begin{lstlisting}[language=C++]
#include "realizations.hpp"
#include <vector>

extern "C" float SinIntegral(float a, float b, float e) {
    float integral = 0.0;
    e = (b - a) / NUM_POINTS;
    for (int i = 1; i < NUM_POINTS; ++i) {
        float x1 = a + i * e;
        float x2 = a + (i + 1) * e;
        integral += 0.5 * e * (sin(x1) + sin(x2));
    }
    return integral;
}

extern "C" int PrimeCount(int a, int b) {
    int count = 0;
    std::vector<int> numbers;
    numbers.reserve(b + 1);
    for (int i = 0; i <= b; ++i) {
        numbers.push_back(i);
    }
    for (int i = 2; i <= b; ++i) {
        if (numbers[i] != 0) {
            if (numbers[i] >= a && numbers[i] <= b) {
                ++count;
            }
            for (int j = i * i; j <= b; j += i) {
                numbers[j] = 0;
            }
        }
    }
    return count;
}
\end{lstlisting}

dynamic\_main.cpp
\begin{lstlisting}[language=C++]
#include "utils.hpp"

int main() {
    const char *pathToLib1 = getenv("PATH_TO_LIB1");
    const char *pathToLib2 = getenv("PATH_TO_LIB2");
    // bash: export PATH_TO_LIB1="/home/kristinab/ubuntu_main/OS_labs/build/lab4/librealization1.so"
    // bash: export PATH_TO_LIB2="/home/kristinab/ubuntu_main/OS_labs/build/lab4/librealization2.so"

    void* libraryHandle = LoadLibrary(pathToLib1);
    SinIntegralFunc SinIntegral = (SinIntegralFunc)dlsym(libraryHandle, "SinIntegral");
    PrimeCountFunc PrimeCount = (PrimeCountFunc)dlsym(libraryHandle, "PrimeCount");

    std::string command;
    while(true) {
        std::cout << "Enter the command (0 - switch realization, e - exit): ";
        std::cin >> command;
        if (command == "e") {
            break;
        } else if (command == "0") {
            std::cout << "Enter the library (1 or 2): ";
            std::cin >> command;
            if (command == "1") {
                libraryHandle = LoadLibrary(pathToLib1);
            } else if (command == "2") {
                libraryHandle = LoadLibrary(pathToLib2);
            } else {
                std::cout << "Invalid library" << std::endl;
            }
            SinIntegral = (SinIntegralFunc)dlsym(libraryHandle, "SinIntegral");
            PrimeCount = (PrimeCountFunc)dlsym(libraryHandle, "PrimeCount");
        } else {
            if (command == "1") {
                std::cout << "SinIntegral function:" << std::endl;
                float arg1, arg2, arg3;
                std::cin >> arg1 >> arg2 >> arg3;
                float result = SinIntegral(arg1, arg2, arg3);
                std::cout << "Result of integral = " << result << std::endl;
            } else if (command == "2") {
                std::cout << "PrimeCount function:" << std::endl;
                int arg1, arg2;
                std::cin >> arg1 >> arg2;
                int result = PrimeCount(arg1, arg2);
                std::cout << "Count of prime numbers = " << result << std::endl;
            } else {
                std::cout << "Invalid command" << std::endl;
            }
        }
    }
    UnloadLibrary(libraryHandle);
    return 0;
}
\end{lstlisting}

static\_main.cpp
\begin{lstlisting}[language=C++]
#include "realizations.hpp"

#include <iostream>

void Task(const std::string& command) {
    if (command == "1") {
        float arg1, arg2, arg3;
        std::cin >> arg1 >> arg2 >> arg3;
        float result = SinIntegral(arg1, arg2, arg3);
        std::cout << "Result of integral = " << result << std::endl;
    } else if (command == "2") {
        int arg1, arg2;
        std::cin >> arg1 >> arg2;
        int result = PrimeCount(arg1, arg2);
        std::cout << "Count of prime numbers = " << result << std::endl;
    } else {
        std::cout << "Invalid command" << std::endl;
    }
}

int main() {
    std::string command;
    while(true) {
        std::cout << "Enter the command (0 - exit): ";
        std::cin >> command;
        if (command == "0") {
            break;
        }
        Task(command);
    }
    return 0;
}
\end{lstlisting}

\newpage
\section{Тесты}
\begin{lstlisting}[language=C++]
#include "gtest/gtest.h"
#include "realizations.hpp"

TEST(fourthLabTest, SinIntegralStaticTest1) {
    float result = SinIntegral(0, PI, 0.01);
    EXPECT_FLOAT_EQ(result, 2);
}

TEST(fourthLabTest, SinIntegralStaticTest2) {
    float result = SinIntegral(0, PI/2, 0.01);
    EXPECT_FLOAT_EQ(result, 1);
}

TEST(fourthLabTest, SinIntegralStaticTest3) {
    float result = SinIntegral(0, PI/3, 0.01);
    EXPECT_FLOAT_EQ(result, 0.49999982);
}

TEST(fourthLabTest, PrimeCountStaticTest) {
    float result = PrimeCount(3, 15);
    EXPECT_FLOAT_EQ(result, 5);
}

int main(int argc, char **argv) {
    testing::InitGoogleTest(&argc, argv);
    return RUN_ALL_TESTS();
}
\end{lstlisting}

\begin{lstlisting}[language=C++]
#include "gtest/gtest.h"
#include "realizations.hpp"

TEST(fourthLabTest, SinIntegralStaticTest1) {
    float result = SinIntegral(0, PI, 0.01);
    EXPECT_FLOAT_EQ(result, 2);
}

TEST(fourthLabTest, SinIntegralStaticTest2) {
    float result = SinIntegral(0, PI/2, 0.01);
    EXPECT_FLOAT_EQ(result, 0.9999997);
}

TEST(fourthLabTest, SinIntegralStaticTest3) {
    float result = SinIntegral(0, PI/3, 0.01);
    EXPECT_FLOAT_EQ(result, 0.49999979);
}

TEST(fourthLabTest, PrimeCountStaticTest) {
    float result = PrimeCount(0, 10);
    EXPECT_FLOAT_EQ(result, 4);
}

int main(int argc, char **argv) {
    testing::InitGoogleTest(&argc, argv);
    return RUN_ALL_TESTS();
}
\end{lstlisting}
\newpage

\section{Консоль}
\begin{verbatim}
kristinab@LAPTOP-SFU9B1F4:~/ubuntu_main/OS_labs/build/lab4$ ./static_main
Enter the command (0 - exit): 1
0 3.14 0.01
Result of integral = 2
Enter the command (0 - exit): 2
3 15
Count of prime numbers = 5
Enter the command (0 - exit): 0
kristinab@LAPTOP-SFU9B1F4:~/ubuntu_main/OS_labs/build/lab4$ ./dynamic_main
Enter the command (0 - switch realization, e - exit): 0
Enter the library (1 or 2): 1
Enter the command (0 - switch realization, e - exit): 1
SinIntegral function:
0 3.14 0.01
Result of integral = 2
Enter the command (0 - switch realization, e - exit): 2
PrimeCount function:
0 10
Count of prime numbers = 4
Enter the command (0 - switch realization, e - exit): 0
Enter the library (1 or 2): 2
Enter the command (0 - switch realization, e - exit): 1
SinIntegral function:
0 3.14 0.01
Result of integral = 2
Enter the command (0 - switch realization, e - exit): 2
PrimeCount function:
0 10
Count of prime numbers = 4
Enter the command (0 - switch realization, e - exit): e
\end{verbatim}

\section{Запуск тестов}
\begin{verbatim}
kristinab@LAPTOP-SFU9B1F4:~/ubuntu_main/OS_labs/build/tests$ ./lab4_test1
[==========] Running 4 tests from 1 test suite.
[----------] Global test environment set-up.
[----------] 4 tests from fourthLabTest
[ RUN      ] fourthLabTest.SinIntegralStaticTest1
[       OK ] fourthLabTest.SinIntegralStaticTest1 (0 ms)
[ RUN      ] fourthLabTest.SinIntegralStaticTest2
[       OK ] fourthLabTest.SinIntegralStaticTest2 (0 ms)
[ RUN      ] fourthLabTest.SinIntegralStaticTest3
[       OK ] fourthLabTest.SinIntegralStaticTest3 (0 ms)
[ RUN      ] fourthLabTest.PrimeCountStaticTest
[       OK ] fourthLabTest.PrimeCountStaticTest (0 ms)
[----------] 4 tests from fourthLabTest (0 ms total)

[----------] Global test environment tear-down
[==========] 4 tests from 1 test suite ran. (0 ms total)
[  PASSED  ] 4 tests.
kristinab@LAPTOP-SFU9B1F4:~/ubuntu_main/OS_labs/build/tests$ ./lab4_test2
[==========] Running 4 tests from 1 test suite.
[----------] Global test environment set-up.
[----------] 4 tests from fourthLabTest
[ RUN      ] fourthLabTest.SinIntegralStaticTest1
[       OK ] fourthLabTest.SinIntegralStaticTest1 (0 ms)
[ RUN      ] fourthLabTest.SinIntegralStaticTest2
[       OK ] fourthLabTest.SinIntegralStaticTest2 (0 ms)
[ RUN      ] fourthLabTest.SinIntegralStaticTest3
[       OK ] fourthLabTest.SinIntegralStaticTest3 (0 ms)
[ RUN      ] fourthLabTest.PrimeCountStaticTest
[       OK ] fourthLabTest.PrimeCountStaticTest (0 ms)
[----------] 4 tests from fourthLabTest (0 ms total)

[----------] Global test environment tear-down
[==========] 4 tests from 1 test suite ran. (0 ms total)
[  PASSED  ] 4 tests.
\end{verbatim}
\newpage

\section{Выводы}

В результате выполнения данной лабораторной работы были созданы динамические библиотеки, которые реализуют функционал в соответствие с вариантом задания на С++. Я приобрела практические навыки в создании программ, которые используют функции динамических библиотек.
\end{document}