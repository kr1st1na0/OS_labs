\documentclass[a4paper, 12pt]{article}
\usepackage{cmap}
\usepackage[12pt]{extsizes}			
\usepackage{mathtext} 				
\usepackage[T2A]{fontenc}			
\usepackage[utf8]{inputenc}			
\usepackage[english,russian]{babel}
\usepackage{setspace}
\singlespacing
\usepackage{amsmath,amsfonts,amssymb,amsthm,mathtools}
\usepackage{fancyhdr}
\usepackage{soulutf8}
\usepackage{euscript}
\usepackage{mathrsfs}
\usepackage{listings}

\usepackage[colorlinks=true, urlcolor=blue, linkcolor=black]{hyperref}

\pagestyle{fancy}
\usepackage{indentfirst}
\usepackage[top=10mm]{geometry}
\rhead{}
\lhead{}
\renewcommand{\headrulewidth}{0mm}
\usepackage{tocloft}
\renewcommand{\cftsecleader}{\cftdotfill{\cftdotsep}}
\usepackage[dvipsnames]{xcolor}

\lstdefinestyle{mystyle}{ 
	keywordstyle=\color{OliveGreen},
	numberstyle=\tiny\color{Gray},
	stringstyle=\color{BurntOrange},
	basicstyle=\ttfamily\footnotesize,
	breakatwhitespace=false,         
	breaklines=true,                 
	captionpos=b,                    
	keepspaces=true,                 
	numbers=left,                    
	numbersep=5pt,                  
	showspaces=false,                
	showstringspaces=false,
	showtabs=false,                  
	tabsize=2
}

\lstset{style=mystyle}

\begin{document}
\thispagestyle{empty}	
\begin{center}
	Московский авиационный институт
	
	(Национальный исследовательский университет)
	
	Факультет "Информационные технологии и прикладная математика"
	
	Кафедра "Вычислительная математика и программирование"
	
\end{center}
\vspace{40ex}
\begin{center}
	\textbf{\large{Лабораторная работа №1 по курсу\linebreak \textquotedblleft Операционные системы\textquotedblright}}
\end{center}
\vspace{35ex}
\begin{flushright}
	\textit{Студент: } Былькова Кристина Алексеевна
	
	\vspace{2ex}
	\textit{Группа: } М8О-208Б-22
	
	\vspace{2ex}
	\textit{Преподаватель: } Миронов Евгений Сергеевич
	
	\vspace{2ex}
	\textit{Вариант: } 16 
	
	\vspace{2ex}
	\textit{Оценка: } \underline{\quad\quad\quad\quad\quad\quad}
	
	 \vspace{2ex}
	\textit{Дата: } \underline{\quad\quad\quad\quad\quad\quad}
	
	\vspace{2ex}
	\textit{Подпись: } \underline{\quad\quad\quad\quad\quad\quad}
	
\end{flushright}

\vspace{5ex}

\begin{vfill}
	\begin{center}
		Москва, 2023
	\end{center}	
\end{vfill}
\newpage


\begingroup
\color{black}
\tableofcontents\newpage
\endgroup

\section{Репозиторий}
\href{https://github.com/kr1st1na0/OS\_labs}{https://github.com/kr1st1na0/OS\_labs}

\section{Цель работы}
Приобретение практических навыков в:
\begin{itemize}
  \item Управлении процессами в ОС
  \item Обеспечении обмена данных между процессами посредством каналов
\end{itemize}

\section{Задание}
Составить и отладить программу на языке Си, осуществляющую работу с процессами и 
взаимодействие между ними в одной из двух операционных систем. В результате работы 
программа (основной процесс) должен создать для решение задачи один или несколько 
дочерних процессов. Взаимодействие между процессами осуществляется через системные 
сигналы/события и/или каналы (pipe).
Необходимо обрабатывать системные ошибки, которые могут возникнуть в результате работы.

\section{Описание работы программы}
Родительский процесс создает дочерний процесс. Первой строкой пользователь в консоль родительского процесса вводит имя файла, которое будет использовано для открытия File с таким 
именем на запись. Родительский и дочерний процесс представлены разными программами. Родительский процесс принимает от пользователя строки произвольной длины и пересылает их в 
pipe1. Процесс child проверяет строки на валидность правилу. Если строка соответствует правилу, то она выводится в стандартный поток вывода дочернего процесса, иначе в pipe2 выводится информация об ошибке. Родительский процесс полученные от child ошибки выводит в стандартный поток вывода.

В ходе выполнения лабораторной работы я использовала следующие системные вызовы:
\begin{itemize}
  \item fork() - создание нового процесса
  \item pipe() - создание канала
  \item dup2() - создание копии файлового дескриптора, используя для нового дескриптора самый маленький свободный номер файлового дескриптора.
  \item execlp() - запуск файла на исполнение
\end{itemize}


\newpage

\section{Исходный код}
utils.hpp
\begin{lstlisting}[language=C++]
#pragma once

#include <iostream>
#include <string>
#include <sstream>
#include <stdlib.h>
#include <unistd.h>
#include <sys/types.h>
#include <ext/stdio_filebuf.h>

void createPipe(int fd[2]);
pid_t createChildProcess();
std::stringstream readFromPipe (int fd);
bool checkString(const std::string &str);
\end{lstlisting}

parent.hpp
\begin{lstlisting}[language=C++]
#pragma once

#include "utils.hpp"

void parentProcess(const char *pathToChild);
\end{lstlisting}

utils.cpp
\begin{lstlisting}[language=C++]
#include "utils.hpp"

void createPipe(int fd[2]) {
    if (pipe(fd) == -1) {
        perror("Couldn't create pipe");
        exit(EXIT_FAILURE);
    }
}

pid_t createChildProcess() {
    pid_t pid = fork();
    if (pid == -1) {
        perror("Couldn't create child process");
        exit(EXIT_FAILURE);
    }
    return pid;
}

std::stringstream readFromPipe (int fd) {
    constexpr int BUFFER_SIZE = 256;
    char buffer[BUFFER_SIZE] = "";
    // char c;
    std::stringstream stream;
    while(true) {       
        int t = read(fd, &buffer, BUFFER_SIZE);
        // int t = read(fd, &c, sizeof(char)); 
        if (t == -1) {
            perror("Couldn't read from pipe");
            exit(EXIT_FAILURE);
        } else if (t > 0) {
            stream << buffer;    
            // stream << c;
        } else {
            return stream;
        }
    }
}

bool checkString(const std::string &str) {
    if (str[str.size() - 1] == '.' || str[str.size() - 1] == ';') {
        return true;
    }
    return false;
}
\end{lstlisting}

child.cpp
\begin{lstlisting}[language=C++]
#include "utils.hpp"

int main(int argc, char *argv[]) {
    if (argc != 2) {
        perror("Not enough arguments");
        exit(EXIT_FAILURE);
    }

    const char *fileName = argv[1];
    std::ofstream fout(fileName);
    if (!fout.is_open()) {
        perror("Couldn't open the file");
        exit(EXIT_FAILURE);
    }
    
    std::string str;
    while (std::getline(std::cin, str)) {
        if (checkString(str)) {
            fout << str << '\n';
        } else {
            std::string error = "ERROR with string: " + str;
            std::cout << error << std::endl;
        }
    }
    
    fout.close();  
    exit(EXIT_SUCCESS);  
}
\end{lstlisting}

parent.cpp
\begin{lstlisting}[language=C++]
#include "parent.hpp"

void parentProcess(const char *pathToChild) {
    std::string fileName;
    getline(std::cin, fileName);

    int fd1[2], fd2[2];
    createPipe(fd1);
    createPipe(fd2);
    
    int pid = createChildProcess();
    if (pid != 0) { // Parent process
        close(fd1[0]);
        close(fd2[1]);
        
        std::string str;
        while (getline(std::cin, str)) {
            str += "\n";
            write(fd1[1], str.c_str(), str.length()); // from str to fd1[1]
        }
        close(fd1[1]);

        std::stringstream output = readFromPipe(fd2[0]);
        while(std::getline(output, str)) {
            std::cout << str << std::endl;
        }
        close(fd2[0]);
    } else { // Child process
        close(fd1[1]);
        close(fd2[0]);

        if (dup2(fd1[0], STDIN_FILENO) == -1 || dup2(fd2[1], STDOUT_FILENO) == -1) {
            perror("Error with dup2");
            exit(EXIT_FAILURE);
        }
        
        if (execlp(pathToChild, pathToChild, fileName.c_str(), nullptr) == -1) { // to child.cpp
            perror("Error with execlp");
            exit(EXIT_FAILURE);
        } 
    }
}
\end{lstlisting}

main.cpp
\begin{lstlisting}[language=C++]
#include "parent.hpp"

int main() {
    parentProcess(getenv("PATH_TO_CHILD"));
    // bash: export PATH_TO_CHILD="/home/kristinab/ubuntu_main/OS_labs/build/lab1/child"
    exit(EXIT_SUCCESS);
}
\end{lstlisting}

\newpage
\section{Тесты}

\begin{lstlisting}[language=C++]
#include <gtest/gtest.h>

#include <filesystem>
#include <memory>
#include <vector>

#include <parent.hpp>

namespace fs = std::filesystem;

void testingProgram(const std::vector<std::string> &input, const std::vector<std::string> &expectedOutput, const std::vector<std::string> &expectedFile) {
    const char *fileName = "file.txt";

    std::stringstream inFile;
    inFile << fileName << std::endl;
    for (std::string line : input) {
        inFile << line << std::endl;
    }

    std::streambuf* oldInBuf = std::cin.rdbuf(inFile.rdbuf()); // выдает старый буфер

    ASSERT_TRUE(fs::exists(getenv("PATH_TO_CHILD")));

    testing::internal::CaptureStdout();

    parentProcess(getenv("PATH_TO_CHILD"));
    std::cin.rdbuf(oldInBuf); // чтобы cin вернулся обратно
    
    std::stringstream errorOut(testing::internal::GetCapturedStdout());
    for(const std::string &expectation : expectedOutput) {
        std::string result;
        getline(errorOut, result);
        EXPECT_EQ(result, expectation);
    }

    std::ifstream fin(fileName);
    if (!fin.is_open()) {
        perror("Couldn't open the file");
        exit(EXIT_FAILURE);
    }
    for (const std::string &expectation : expectedFile) {
        std::string result;
        getline(fin, result);
        EXPECT_EQ(result, expectation);
    }
    fin.close();
}

TEST(firstLabTests, emptyTest) {
    std::vector<std::string> input = {};

    std::vector<std::string> expectedOutput = {};

    std::vector<std::string> expectedFile = {};

    testingProgram(input, expectedOutput, expectedFile);
}

TEST(firstLabTests, simpleTest) {
    std::vector<std::string> input = {
        "No,",
        "you'll never be alone.",
        "When darkness comes;",
        "I'll light the night with stars",
        "Hear my whispers in the dark!"
    };

    std::vector<std::string> expectedOutput = {
        "ERROR with string: No,",
        "ERROR with string: I'll light the night with stars",
        "ERROR with string: Hear my whispers in the dark!"
    };

    std::vector<std::string> expectedFile = {
        "you'll never be alone.",
        "When darkness comes;"
    };

    testingProgram(input, expectedOutput, expectedFile);
}

TEST(firstLabTests, aQuedaTest) {
    std::vector<std::string> input = {
        "A QUEDA:",
        "E venha ver os deslizes que eu vou cometer;",
        "E venha ver os amigos que eu vou perder;",
        "Não tô cobrando entrada, vem ver o show na faixa.",
        "Hoje tem open bar pra ver minha desgraça."
    };

    std::vector<std::string> expectedOutput = {
        "ERROR with string: A QUEDA:"
    };

    std::vector<std::string> expectedFile = {
        "E venha ver os deslizes que eu vou cometer;",
        "E venha ver os amigos que eu vou perder;",
        "Não tô cobrando entrada, vem ver o show na faixa.",
        "Hoje tem open bar pra ver minha desgraça."
    };

    testingProgram(input, expectedOutput, expectedFile);
}

TEST(firstLabTests, anotherTest) {
    std::vector<std::string> input = {
        "But I set fire to the rain.",
        "Watched it pour as- I touched your- face-",
        "Well, it burned while I cried!!!!!!!!",
        "Cause I heard it screamin' out your name;",
        "Your name."
    };

    std::vector<std::string> expectedOutput = {
        "ERROR with string: Watched it pour as- I touched your- face-",
        "ERROR with string: Well, it burned while I cried!!!!!!!!"
    };

    std::vector<std::string> expectedFile = {
        "But I set fire to the rain.",
        "Cause I heard it screamin' out your name;",
        "Your name."
    };    

    testingProgram(input, expectedOutput, expectedFile);
}

int main(int argc, char *argv[]) {
    std::cout << getenv("PATH_TO_CHILD") << std::endl;
    
    testing::InitGoogleTest(&argc, argv);
    return RUN_ALL_TESTS();
}
\end{lstlisting}

\newpage
\section{Демонстрация работы программы}

\begin{verbatim}
kristinab@LAPTOP-SFU9B1F4:~/ubuntu_main/OS_labs/build/lab1$ ./child file.txt
No,
you'll never be alone.
When darkness comes;
I'll light the night with stars
Hear my whispers in the dark!
ERROR with string: I'll light the night with stars
ERROR with string: No,
ERROR with string: Hear my whispers in the dark!
kristinab@LAPTOP-SFU9B1F4:~/ubuntu_main/OS_labs/build/lab1$ ./child file.txt
A QUEDA:
E venha ver os deslizes que eu vou cometer;
E venha ver os amigos que eu vou perder;
Não tô cobrando entrada, vem ver o show na faixa.
Hoje tem open bar pra ver minha desgraça.
ERROR with string: A QUEDA:
kristinab@LAPTOP-SFU9B1F4:~/ubuntu_main/OS_labs/build/lab1$ ./child file.txt
But I set fire to the rain.
Watched it pour as- I touched your- face-
Well, it burned while I cried!!!!!!!!
Cause I heard it screamin' out your name;
Your name.
ERROR with string: Watched it pour as- I touched your- face-
ERROR with string: Well, it burned while I cried!!!!!!!!
\end{verbatim}

\section{Запуск тестов}
\begin{verbatim}
kristinab@LAPTOP-SFU9B1F4:~/ubuntu_main/OS_labs/build/tests$ ./lab1_test
/home/kristinab/ubuntu_main/OS_labs/build/lab1/child
[==========] Running 4 tests from 1 test suite.
[----------] Global test environment set-up.
[----------] 4 tests from firstLabTests
[ RUN      ] firstLabTests.emptyTest
[       OK ] firstLabTests.emptyTest (7 ms)
[ RUN      ] firstLabTests.simpleTest
[       OK ] firstLabTests.simpleTest (1 ms)
[ RUN      ] firstLabTests.aQuedaTest
[       OK ] firstLabTests.aQuedaTest (0 ms)
[ RUN      ] firstLabTests.anotherTest
[       OK ] firstLabTests.anotherTest (1 ms)
[----------] 4 tests from firstLabTests (11 ms total)

[----------] Global test environment tear-down
[==========] 4 tests from 1 test suite ran. (11 ms total)
[  PASSED  ] 4 tests.
\end{verbatim}
\newpage
\section{Выводы}

В результате выполнения данной лабораторной работы была написана программа на языке С++, осуществляющая работу с процессами и 
взаимодействие между ними. Я приобрела практические навыки в управлении процессами в ОС и обеспечении обмена данных между процессами посредством каналов.
\end{document}